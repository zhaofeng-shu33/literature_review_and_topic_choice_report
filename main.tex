\documentclass{ctexart}
\usepackage{hyperref}
\title{文献综述与选题书面报告 \\ \normalsize{论文题目:基于信息论度量的社群发现理论与算法研究 }
}
\author{赵丰 \\ 学号: 2017310711}
\begin{document}
\maketitle
\section{选题背景及其意义}
在网络科学中,复杂的网络结构往往存在一定的社群结构 \cite{fortunato2010community} 。社群结构是指
具有相同特征的节点集合,
比如社交网络中的圈子、流行病网络中集中爆发的区域。以及在自然界中的群落以及微观世界网络中的大分子
蛋白质。社群发现是通过一定的方法寻找网络中特定的社群结构。在自然科学研究中有助于发现特定的结构。而在
技术领域则是实现某项技术目标的重要一环。比如在分布式计算中划分子任务使得不同的子任务之间有尽可能
小的耦合,在推荐系统设计中划分用户社群以实现精准推荐。社群发现的实现离不开特定的算法,而好的社群发现算法
需要有特定的理论支撑,以便助其在各领域实现广泛应用。

社群发现算法需要适合特定的场景。考虑到有一些问题中社群之间具有复杂的相关关系,从而构成一定的层次结构。如
在社交网络中根据兴趣对目标人群的划分则有大的尺度和精细的尺度多个维度。普通的社群发现算法无法适应社群的层次
结构划分的需要。基于特定理论的指导,研究新的分层发现算法对解决某些特定应用场景的问题具有重要的实用价值。

社群发现有大量的算法可供使用。由于对社群发现问题的概率模型的理论
极限和部分经典算法的信息学含义缺少足够的认识,这些启发式算法在消耗大量计算资源的同时效果也
不一定能得到保障。为解决该问题,需要对社群发现的理论模型开展深入研究。
社群发现理论的研究是一个涉及信息论、图论和概率论等多学科交叉融合的领域。目前基于随机块模型的理论研究思路
取得了一定的突破,随机块模型提供了比较不同的社群发现算法的标准化人工生成的数据集。此外,从理论层面研究随机块模型下算法的误差可以
指导算法的设计。
目前的研究给出了若干
在随机块模型下具有理论保证的几类算法,通过对这些算法做出近似和调整,可以适用于实际的社群发现问题。随机块模型的研究
还可以用于解决和社群发现密切相关的问题,比如根据若干次民意调查的结果预测选民的政治倾向等。

在分析具有图结构的数据时,通常每个节点会有一些额外的信息可供使用,比如每个节点的特征属性。比如在社交网络中利用
用户间的交互关系和每一用户自身的属性对用户群体进行划分。
如何利用这些节点的观测值提高
社群发现的准确率也是近年来研究的一个热点。已经涌现了大量的算法可以利用节点和图的信息进行社群发现,但在这方面缺少
针对误差速率的理论分析。这方面的理论分析可以指导特征数量的选取,以提高数据的使用效率,避免浪费。此外,这一部分的研究
还可以用在其他具有相似数学模型的领域。比如有相关关系的多个数据源的信息压缩问题。

信息论的度量如熵、互信息等最初是在通信领域用于度量编码和信道传输的理论极限,逐渐被
推广到了机器学习、网络科学等领域。信息论的度量如 KL 散度、互信息等 可以作为算法的评价指标,
以及出现在理论模型中误差的理论极限中。利用信息论中误差指数的研究方法,可以有效地分析最大似然算法在随机块模型上的
恢复误差。


\section{国内外研究动态}
如同选题背景中所介绍的,本课题主要包含层次发现算法、随机块模型的精确恢复问题及有额外信息的随机块模型三个方面。
下面将从这三个方面分别阐述国内外研究的动态。
\subsection{层次发现算法}
最早的层次发现算法是系统聚类法 \cite{slink},即根据欧式度量每次聚合两个节点形成聚类树。
近年来基于各领域的学科知识出现了大量新的层次发现算法,比如基于密度估计的方法 \cite{hde},
贝叶斯聚类方法 \cite{bhc}、基于凸优化的方法 \cite{hocking2011clusterpath}、基于图论和离散数学的方法 \cite{dasgupta2016cost}
、基于双曲几何度量的方法 \cite{hyperbolic} 等。
\section{课题研究内容}
\section{研究方案}
\section{工作特色及难点}
\section{预期成果和可能的创新点}
\section{论文工作的总体安排}
\bibliographystyle{unsrt}
\bibliography{exportlist.bib}
\end{document}
